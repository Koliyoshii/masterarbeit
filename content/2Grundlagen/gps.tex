\section{GPS}
\label{tracking-gps-tracking}
Im Außenbereich wird bei AR auch GPS für das Tracking herangezogen. Dabei sind Positionsabweichungen von bis zu 10 Metern möglich, sodass eine genaue Bestimmung der extrinsischen Kameraparameter beeinträchtigt wird. Um die Tracking-Genauigkeit zu erhöhen, gibt es mehrere Methoden. \textit{DGPS (engl. Differential GPS)} verbessert GPS-Signale, indem es ein Korrektursignal durch eine ortsfeste Referenzstation mit bekannter Lokalisierung berechnet. Da es in Deutschland lediglich acht solcher Stationen gibt und einige Anbieter nur kommerziell die Daten bereitstellen, ist diese Methode nicht für diese Arbeit geeignet\footnote{https://www.heise.de/newsticker/meldung/Differential-GPS-und-WLAN-RTT-Praezise-Ortung-mit-Android-P-4046935.html} \footnote{Liste von DGPS-Sendern: https://www.ndblist.info/datamodes/worldDGPSdatabase.pdf}. Eine weitere bekannte Möglichkeit bietet \textit{SBAS (engl. Satellite Based Augmentation System)}, bei dem mehrere geostationäre Satelliten das GPS Signal auf bis zu einem Meter Genauigkeit zu verbessern \cite*{doerner}.

Platinsky und seine Koautoren\cite{platinsky} erstellen für ein besseres Tracking bei fehlender GPS Genauigkeit ein 3D-Modell der Umgebung. Bei der anschließenden AR Nutzung in diesem Gebiet wird auf dem Smartphone SLAM betrieben. Die Daten vom Smartphone werden mit der 3D-Karte verglichen, um ein genaueres Tracking durchzuführen. Ein ähnliches System wäre für die Anwendung in dieser Arbeit denkbar, da eine große Datenmenge von Bildern des Geländes vorhanden ist. Über Structure from Motion Methoden, kann mit den Bildern eine große 3D-Karte erstellt werden.