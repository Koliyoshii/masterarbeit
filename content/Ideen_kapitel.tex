\section{Hintergrund der Arbeit}
Ein kurzes Kapitel über die vorangegangenen Projekte, Verweise auf Projekt NISABA etc.

Welche Gebäude sind vorhanden? Bestandsaufnahme vorhandener Gebäude. 

\section{Aufgabenstellung}
Die Aufgabe besteht darin die im vorhergegangenen Arbeiten entstandenen 3D Modelle des SABA Gebäudes, des Lyautley- und des Mangin-Geländes mithilfe einer AR Applikation zu visualisieren. Ziel ist es eine App zu entwickeln, die eine hohe Immersion bietet. Daher wird zunächst der Begriff der Immersion in AR-Anwendungen definiert. Die Zielgruppe sind Bewohner der Stadt Villingen und interessierte Personen wie Touristen, die mehr über die Geschichte der Stadt Villingen erfahren möchten. Die Anwendung wird vor Ort genutzt, da die Geäude im Größenverhältnis 1:1 dargestellt werden. So hat der Nutzer ein direktes Verständnis über die Größenverhältnisse der Gebäude. Die 3D Modelle sollen an das Umgebungslicht und der herrschenden Wetterbedingungen angepasst sein. Das bedeutet, dass bei harten Licht die Modelle auch harte Schattenkanten werfen. Bei regnerischen Wetter soll die Fassade dunkler wirken, diffuse Schatten werfen und eventuell Spiegelungen darstellen.

\subsection{Fragestellungen / Problemstellungen}
Was sind die Probleme in der Entwicklung von AR Anwendungen im Freien?

- GPS ungenauigkeit

- größere Entfernung -> genaueres Tracking benötigt

Realistisches Umgebungsslicht

Wetterphänome und Darstellung in Computergraphik

\section{Forschungsstand}
\subsection{Ähnliche Arbeiten}
Outdoor AR Projekte

Wie wurden die Probleme gelöst?

\section{Tracking}
Definition, was ist mit dem Begriff gemeint?
\subsection{Koordinatensysteme}
\subsection{Kamera-basiertes Tracking}
\subsection{Marker Tracking}
\subsection{Marker-less Tracking}
\subsection{Algorithmen zur Erkennnung von Merkmalen}
\subsubsection{SIFT}
\subsubsection{SURF}
\subsubsection{ORB}
\subsubsection{Merkmalserkennung in AR Foundation / AR Core}
\subsection{SLAM}
Mathematische Definition des SLAM Problems und bekannte Arbeiten zur Lösung des Problems.
\subsection{SLAM in ARFoundation / AR Core}
\subsection{GPS Tracking}


\section{Computergraphik Shader}
Darstellung von Wetterphänomenen (hartes/diffuses Licht, Spiegelungen, Raytracing?)
- Raytracing in mobiltelefon in Unity machbar?
\subsection{Rendering Pipeline}
\subsection{Darstellung von Wetterphanomenen in der Computergraphik}
\subsection{Erkennung der Lichtsituation in AR Foundation / AR Core}
\subsection{Spiegelungen, Raytracing(?)}
Wie werden Spiegelungen erzeugt?

Raytracing, auch in Smartphones schon möglich?

\section{Entwicklung}
\subsection{Methodiken in der Planung}
UML, Agile Softwareentwicklung

\subsection{Frameworks}
\subsection{(Vergleich vorhandener Frameworks)}
Was gibt es auf dem Markt (AR Foundation, AR Core, AR Kit, Vuforia, Kudan, Wiktude), Vorteile/Nachteile und geeignetes Framework für die Entwicklung dieser App.
\subsection{Entwicklungsumgebung}
Ein paar beschreibende Worte über AR Foundation und Unity.

Welche Sprache wird genutzt?

Wie erfolgt die Entwicklung? Code schreiben, Debugging, Smartphone anschließen

Was muss für die Entwicklung installiert werden?

\subsection{Aufbau der Anwendung}
Beschreibung des Aufbaus der App, also die UML Grafik

\section{Nutzertest bzw. Evaluierung}
\subsection{Aufbau des Tests}
\subsection{Durchführung}
\subsection{Ergebnisse/Evaluierung}

\section{Fazit der Arbeit}
\section{Ausblick}