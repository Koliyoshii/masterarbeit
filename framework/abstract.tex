\addcontentsline{toc}{section}{Abstract} % \abstractname is the default name, you can customize it
%\thispagestyle{plain}
\begin{center}
    \Large
    \textbf{Abstract}
\end{center}

\noindent
Durch das Projekt NISABA\footnote{https://nisaba.dm.hs-furtwangen.de/} und der Veranstaltung Bildverarbeitung und Computergrafik im Studiengang Medieninformatik Master im Sommersemester 2021 sind 3D-Modelle der Kasernengebäude des Lyautey- und Mangin-Geländes entstanden. Um die fertigen 3D Modelle für jeden zugänglich und auf moderner Weise präsentieren zu können, wird in dieser Master-Arbeit eine Augmented Reality Anwendung für Smartphone und Tablet Geräte entwickelt.

Diese Forschungsarbeit beschäftigt sich mit der Entwicklung dieser Augmented Reality \acrshort{ar} Anwendung. Durch genaues Tracking und eine realistische Darstellung der Gebäude durch wetterspezifische Belichtung, Schattierung und Spiegelung soll eine hohe Immersion erschaffen werden. In der Anwendung werden die 3D-Objekte in der Größendarstellung 1:1 über das Videobild der Smartphone Kamera gelegt und der Nutzer kann vor Ort das Gebäude in Echtzeit ein- und ausblenden lassen. Bekannte Beispiele sind Pokemon GO\footnote{https://pokemongolive.com/de/} oder Googles Holo Lens\footnote{https://www.microsoft.com/de-de/hololens}. In der Fachsprache werden diese Anwendungen \textit{world-scale Augmented Reality} genannt.

Eine wesentliche Rolle bei der Immersion in AR spielt das Tracking, bei dem kontinuierlich die Positions- und Orientierungsdaten des Endgeräts erfasst werden. Es gibt mehrere Verfahren, um die Position und Orientierung der Kamera relativ zur Umgebung zu bestimmen. Im begrenzten Räumen ist das Tracking durch die einheitliche Belichtung und vorhandenen Kanten und Flächen mit kamerabasierten Tracking Methoden gut umsetzbar. Im Freien kann das Tracking je nach Umgebung Probleme bereiten. Schwierigkeiten entstehen durch hohe Dynamik und dadurch, dass die Umgebung unbegrenzt ist. Hinzu kommt, dass durch die große Distanz zwischen Kamera und virtuellem Objekt die Platzierung des 3D-Modells bereits durch kleine Bewegung der Kamera stark von der korrekten Position abweicht. Das Objekt erscheint nicht homogen in der realen Welt, wodurch die Immersion beeinträchtigt wird.
%Was sind die Probleme bei Outdoor Augmented Reality? Recherche

Klassischerweise wird bei AR im Freien GPS und die Neigungs- und Beschleunigungssensoren der Smartphones genutzt, um die Kameraposition und -orientierung zu ermitteln. Wie Platinsky et al.\cite*{platinsky} bereits erörtert ist die GPS Lokalisierung insbesondere in Städten mit Störfaktoren wie Gebäuden oder Vegetation ungenau. Deshalb wird in dieser Arbeit auf Methoden eingegangen, um die Genauigkeit der Position und der Orientierung der Kamera im Freien zu verbessern.

Um die Immersion bei der Nutzung der App zu steigern, werden die Wetterbedingungen bei der Darstellung der 3D-Modelle berücksichtigt. So sollen die Fassaden bei regnerischem Wetter dunkler und feucht bzw. nass dargestellt werden. Die Schattenwürfe sollen sich anpassen, indem bei hartem Licht (z.B. durch starke Sonneneinstrahlung) auch harte Schatten und bei weichem Licht (z.B. bei einer dichten Wolkendecke) weiche Schatten dargestellt werden.

Fazit: