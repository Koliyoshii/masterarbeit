% ##################################################
% ENCODING
% ##################################################
\usepackage{cmap}               % PDF character encoding
\usepackage[T1]{fontenc}        % 8-bit font encoding
\usepackage[utf8]{inputenc}     % UTF-8 input encoding

% ##################################################
% LANGUAGE
% ##################################################
\usepackage[ngerman]{babel} % Uncomment this if you write in German
\usepackage{csquotes}         % Package for better quotations
\hyphenation{}                % Here you can define custom hyphenations
\usepackage{blindtext}        % This package can usually be deleted as this is only used to display some blindtext

% ##################################################
% DOCUMENT VARIABLES
% ##################################################
% Personal data
\newcommand{\docAuthor}{Oliver Kusch}
\newcommand{\docMatriculationNumber}{268513}
\newcommand{\docStudyProgram}{Medieninformatik Master}
\newcommand{\docStudyFaculty}{Digitale Medien}
\newcommand{\docStudyDegree}{Master}
\newcommand{\docStreetName}{Erbsenlachen 6}
\newcommand{\docPostalCode}{78050}
\newcommand{\docCity}{Villingen-Schwenningen}
\newcommand{\docEmail}{oliver.kusch@hs-furtwangen.de}

% Document data
\newcommand{\docTitle}{Entwicklung einer Augmented Reality App zur Visualisierung historischer Gebäude der Stadt Villingen} % This is the main title of this document (thesis)
\newcommand{\docSubTitle}{}            % leave empty to remove
\newcommand{\docType}{Thesis}                            % Usually this will be 'Thesis"
\newcommand{\docSupervisor}{Prof. Dr. Uwe Hahne}             % Your supervisor
\newcommand{\docCoSupervisor}{Prof. Dr. Thomas Schneider}        % Your co-supervisor
\newcommand{\docDeadline}{31.08.2022}                      % This is the date when you submit the document

% ##################################################
% BIBLIOGRAPHY 
% ##################################################
\usepackage[backend=bibtex]{biblatex}
\addbibresource{literatur/library.bib}

\usepackage[acronym, toc, nogroupskip, nopostdot, nonumberlist]{glossaries}
\usepackage[acronym]{glossaries}
%\makeglossaries

%Acronym AR und Tracking
\newacronym{ar}{AR}{Augmented Reality}
\newacronym[]{vr}{VR}{Virtual Reality}
\newacronym[]{mr}{MR}{Mixed Reality}
\newacronym[]{ve}{VE}{Virtual Environments}
\newacronym[]{imu}{IMU}{Inertial Measurement Units}
\newacronym[]{dof}{DOF}{Degrees of Freedom}
\newacronym[]{fast}{FAST}{Features from accelerated segment test}
\newacronym[]{sift}{SIFT}{Scale Invariant Feature Transform}
\newacronym[]{surf}{SURF}{Speeded Up Robust Features}
\newacronym[]{orb}{ORB}{Oriented Fast and Rotated Brief}
\newacronym{slam}{SLAM}{Simultaneous Localization and Mapping}
\newacronym{ptam}{PTAM}{Parallel Tracking and Mapping}
\newacronym{uml}{UML}{Unified Modeling Language}
\newacronym{rest}{REST}{Respresentational State Transfer}
\newacronym{json}{JSON}{JavaScript Object Notation}

%Acronyms App
\newacronym{sdk}{SDK}{Software Develpoment Kit}

\newacronym{Materials}{Materials}{Materials affektieren das Aussehen eines 3D Objekts mit einer oder mehrerer Texturen, Farben oder den Eigenschaften von Reflexionen}
\newacronym{grafik-rendering-pipeline}{GRP}{Grafik-Rendering-Pipeline}
\newacronym{frames-per-second}{FPS}{Frames Pro Sekunde}
\newacronym{cpu}{CPU}{Central Processing Unit}
\newacronym{gpu}{GPU}{Graphics Processing Unit}

\makenoidxglossaries


% ##################################################
% PDF SETTINGS
% ##################################################
\usepackage[
    colorlinks=true,
    linkcolor=black,
    citecolor=black,
    filecolor=black,
    urlcolor=black,
    bookmarks=true,
    bookmarksopen=true,
    bookmarksopenlevel=3,
    bookmarksnumbered,
    plainpages=false,
    pdfpagelabels=true,
    hyperfootnotes,
    pdftitle ={\docTitle},
    pdfauthor={\docAuthor},
    pdfcreator={\docAuthor}
]{hyperref}

% ##################################################
% FONTS AND SPACING
% ##################################################

\renewcommand{\baselinestretch}{1.5} % default 1.5 spacing
\raggedbottom                        % don't stretch spacing to fit page length
\usepackage{lmodern}                 % Allow font sizes at arbitrary sizes
\usepackage{microtype}               % Bessere Worttrennung. Wirkt overful hbox entgegen
\usepackage{textgreek}               % Allow alpha, beat, gamma ... in text without Mathmode

% ##################################################
% PAGE FORMATTING
% ##################################################
% Page layout, see http://mirrors.ctan.org/macros/latex/contrib/geometry/geometry.pdf #3 Page geometry
\usepackage[
    bindingoffset=1.5cm, % Specify an offset if you plan to bind the document
    inner=2.5cm,         % Left Spacing
    outer=2.5cm,         % Right spacing
    top=3cm,             % Top spacing
    bottom=2cm,          % Bottom spacing
    twoside              % Delete this if you don't want to use double-sided pages
]{geometry}

% Page header
\usepackage{fancyhdr}
\fancypagestyle{scrheadings}
{
  \fancyhead[EL]{\thepage}% gerade Seiten, links
  \fancyhead[ER]{\nouppercase{\leftmark}}% gerade Seiten, rechts
  \fancyhead[OL]{\nouppercase{\leftmark}}% ungerade Seiten, links
  \fancyhead[OR]{\thepage}% ungerade Seiten, rechts
  \fancyfoot[C]{}
}
\setlength{\headheight}{18pt}           % set headheight to 18pt
\addtolength{\skip\footins}{0.5cm}      % spacing at the end of page (footnotes)

%\usepackage[
 %   headsepline,                        % seperator line beneath page header on normal pages
 %   plainheadsepline                    % seperator line beneath page header on pages like ToC
%]{scrlayer-scrpage}
%\clearpairofpagestyles                  % clear default settings
%\addtokomafont{pagehead}{\normalfont}   % use normal font for page header
%\ohead*{\thepage}                       % page number
%\ihead{\headmark}                       % chapter name
%\automark[section]{section}             


\usepackage[bottom]{footmisc}           % Always place footer at end of page

\usepackage{multicol}                   % Format text in multiple columns
\usepackage{emptypage}                  % Support for empty pages via \cleardoublepage
\usepackage{lscape}                     % Support for landscape pages
\usepackage{titlesec}
\usepackage{hyperref}

\titleclass{\subsubsubsection}{straight}[\subsection]

\newcounter{subsubsubsection}[subsubsection]
\renewcommand\thesubsubsubsection{\thesubsubsection.\arabic{subsubsubsection}}
\renewcommand\theparagraph{\thesubsubsubsection.\arabic{paragraph}} % optional; useful if paragraphs are to be numbered

\titleformat{\subsubsubsection}
  {\normalfont\normalsize\bfseries}{\thesubsubsubsection}{1em}{}
\titlespacing*{\subsubsubsection}
{0pt}{3.25ex plus 1ex minus .2ex}{1.5ex plus .2ex}

\makeatletter
\renewcommand\paragraph{\@startsection{paragraph}{5}{\z@}%
  {3.25ex \@plus1ex \@minus.2ex}%
  {-1em}%
  {\normalfont\normalsize\bfseries}}
\renewcommand\subparagraph{\@startsection{subparagraph}{6}{\parindent}%
  {3.25ex \@plus1ex \@minus .2ex}%
  {-1em}%
  {\normalfont\normalsize\bfseries}}
\def\toclevel@subsubsubsection{4}
\def\toclevel@paragraph{5}
\def\toclevel@paragraph{6}
\def\l@subsubsubsection{\@dottedtocline{4}{7em}{4em}}
\def\l@paragraph{\@dottedtocline{5}{10em}{5em}}
\def\l@subparagraph{\@dottedtocline{6}{14em}{6em}}
\makeatother

\setcounter{secnumdepth}{4}
\setcounter{tocdepth}{4}

% ##################################################
% IMAGES AND FIGURES
% ##################################################
\usepackage{graphicx}                   % support for including images
\graphicspath{{img/}}                   % default path
\usepackage{subfig}

% simple numbering without chapter
\renewcommand{\thefigure}{\arabic{figure}}

% ##################################################
% TikZ | Support for drawing diagrams
% ##################################################
\usepackage{tikz}
\usepackage{float}
\usetikzlibrary{shapes.geometric, arrows, positioning, decorations.pathreplacing, calc}
\tikzstyle{box} = [rectangle, minimum width=3cm, minimum height=1cm, text centered, draw=black, fill=orange!30]
\tikzstyle{mainbox} = [rectangle, minimum width=3cm, minimum height=1cm, text centered, draw=black, fill=green!30]
\tikzstyle{plainbox} = [rectangle, minimum width=3cm, minimum height=1cm, text centered, draw=black, fill=white,thick]
\tikzstyle{border} = [rectangle, minimum width=3cm, minimum height=1cm, text centered, draw=black]
\tikzstyle{arrow} = [thick,->,>=stealth]

\pgfdeclarelayer{foreground}
\pgfdeclarelayer{background}
\pgfsetlayers{background,main,foreground}

% ##################################################
% MATH
% ##################################################
\usepackage{amstext}
\usepackage{amsmath}

\usepackage{tabularx}
\newenvironment{conditions*}
  {\par\vspace{\abovedisplayskip}\noindent
   \tabularx{\columnwidth}{>{$}l<{$} @{${}={}$} >{\raggedright\arraybackslash}X}}
  {\endtabularx\par\vspace{\belowdisplayskip}}

% ##################################################
% SUPPORT FOR DIRECTORY TREE RENDERING
% ##################################################
\usepackage{dirtree}

% ##################################################
% SUPPORT FOR CODE LISTINGS
% ##################################################
\usepackage{listings}
\lstset{numbers=left}
\usepackage{color}
\usepackage{xcolor}

% \definecolor{backcolour}{rgb}{0.95,0.95,0.95} % Custom color for code background
% \definecolor{codegreen}{rgb}{0,0.6,0}         % Custom color for comments in your code
% \definecolor{codegray}{rgb}{0.5,0.5,0.5}      % Custom color for numbers in your code
% \definecolor{codepurple}{rgb}{0.58,0,0.82}    % Currently unused

% \renewcommand{\lstlistingname}{Code Snippet} % Here you can change the code listing name


\definecolor{codegreen}{rgb}{0,0.6,0}
\definecolor{codegray}{rgb}{0.5,0.5,0.5}
\definecolor{codepurple}{rgb}{0.58,0,0.82}
\definecolor{backcolour}{rgb}{0.95,0.95,0.92}

\lstdefinestyle{mystyle}{
    backgroundcolor=\color{backcolour},   
    commentstyle=\color{codegreen},
    keywordstyle=\color{magenta},
    numberstyle=\tiny\color{codegray},
    stringstyle=\color{codepurple},
    basicstyle=\ttfamily\footnotesize,
    breakatwhitespace=false,         
    breaklines=true,                 
    captionpos=b,                    
    keepspaces=true,                 
    numbers=left,                    
    numbersep=5pt,                  
    showspaces=false,                
    showstringspaces=false,
    showtabs=false,                  
    tabsize=2
}
\lstset{style=mystyle}

% \lstdefinestyle{normal}{
%     backgroundcolor=\color{backcolour},
%     commentstyle=\color{codegreen},
%     keywordstyle=\color{magenta},
%     numberstyle=\tiny\color{codegray},
%     stringstyle=\color{orange},
%     basicstyle=\ttfamily\small,
%     breakatwhitespace=false,
%     breaklines=true,
%     captionpos=b,
%     keepspaces=true,
%     numbers=left,
%     numbersep=5pt,
%     showspaces=false,
%     showstringspaces=false,
%     showtabs=false,
%     tabsize=2,
%     rulecolor=\color{black},
%     frame=L,
%     xleftmargin=6px
% }
% \lstset{style=normal}