% ##################################################
% ENCODING
% ##################################################
\usepackage{cmap}               % PDF character encoding
\usepackage[T1]{fontenc}        % 8-bit font encoding
\usepackage[utf8]{inputenc}     % UTF-8 input encoding

% ##################################################
% LANGUAGE
% ##################################################
\usepackage[ngerman]{babel} % Uncomment this if you write in German
\usepackage{csquotes}         % Package for better quotations
\hyphenation{}                % Here you can define custom hyphenations
\usepackage{blindtext}        % This package can usually be deleted as this is only used to display some blindtext

% ##################################################
% DOCUMENT VARIABLES
% ##################################################
% Personal data
\newcommand{\docAuthor}{Oliver Kusch}
\newcommand{\docMatriculationNumber}{268513}
\newcommand{\docStudyProgram}{Medieninformatik Master}
\newcommand{\docStudyFaculty}{Digitale Medien}
\newcommand{\docStudyDegree}{Master}
\newcommand{\docStreetName}{Erbsenlachen 6}
\newcommand{\docPostalCode}{78050}
\newcommand{\docCity}{Villingen-Schwenningen}
\newcommand{\docEmail}{oliver.kusch@hs-furtwangen.de}

% Document data
\newcommand{\docTitle}{Entwicklung einer Augmented Reality App zur Visualisierung historischer Gebäude der Stadt Villingen} % This is the main title of this document (thesis)
\newcommand{\docSubTitle}{}            % leave empty to remove
\newcommand{\docType}{Thesis}                            % Usually this will be 'Thesis"
\newcommand{\docSupervisor}{Prof. Dr. Uwe Hahne}             % Your supervisor
\newcommand{\docCoSupervisor}{Prof. Dr. Thomas Schneider}        % Your co-supervisor
\newcommand{\docDeadline}{31.08.2022}                      % This is the date when you submit the document

% ##################################################
% BIBLIOGRAPHY 
% ##################################################
\usepackage[backend=bibtex]{biblatex}
\addbibresource{literatur/library.bib}

\usepackage[acronym, toc, nogroupskip, nopostdot, nonumberlist]{glossaries}
\usepackage[acronym]{glossaries}
%\makeglossaries


\newacronym{ar}{AR}{Augmented Reality}
\newacronym[]{vr}{VR}{Virtual Reality}
\newacronym[]{mr}{MR}{Mixed Reality}
\newacronym[]{ve}{VE}{Virtual Environments}
\newacronym[]{imu}{IMU}{Inertial Measurement Units}
\newacronym[]{dof}{DOF}{Degrees of Freedom}
\newacronym[]{sift}{SIFT}{Scale Invariant Feature Transform}
\newacronym[]{surf}{SURF}{Speeded Up Robust Features}
\newacronym{slam}{SLAM}{Simultaneous Localization and Mapping}
\newacronym{Materials}{Materials}{Materials affektieren das Aussehen eines 3D Objekts mit einer oder mehrerer Texturen, Farben oder den Eigenschaften von Reflexionen}
\newacronym{grafik-rendering-pipeline}{GRP}{Grafik-Rendering-Pipeline}
\newacronym{frames-per-second}{FPS}{Frames Pro Sekunde}
\newacronym{cpu}{CPU}{Central Processing Unit}
\newacronym{gpu}{GPU}{Graphics Processing Unit}

\makenoidxglossaries


% ##################################################
% PDF SETTINGS
% ##################################################
\usepackage[
    colorlinks=true,
    linkcolor=black,
    citecolor=black,
    filecolor=black,
    urlcolor=black,
    bookmarks=true,
    bookmarksopen=true,
    bookmarksopenlevel=3,
    bookmarksnumbered,
    plainpages=false,
    pdfpagelabels=true,
    hyperfootnotes,
    pdftitle ={\docTitle},
    pdfauthor={\docAuthor},
    pdfcreator={\docAuthor}
]{hyperref}

% ##################################################
% FONTS AND SPACING
% ##################################################

\renewcommand{\baselinestretch}{1.5} % default 1.5 spacing
\raggedbottom                        % don't stretch spacing to fit page length
\usepackage{lmodern}                 % Allow font sizes at arbitrary sizes

% ##################################################
% PAGE FORMATTING
% ##################################################
% Page layout, see http://mirrors.ctan.org/macros/latex/contrib/geometry/geometry.pdf #3 Page geometry
\usepackage[
    bindingoffset=1.5cm, % Specify an offset if you plan to bind the document
    inner=2.5cm,         % Left Spacing
    outer=2.5cm,         % Right spacing
    top=3cm,             % Top spacing
    bottom=2cm,          % Bottom spacing
    twoside              % Delete this if you don't want to use double-sided pages
]{geometry}

% Page header
\usepackage[
    headsepline,                        % seperator line beneath page header on normal pages
    plainheadsepline                    % seperator line beneath page header on pages like ToC
]{scrlayer-scrpage}
\clearpairofpagestyles                  % clear default settings
\addtokomafont{pagehead}{\normalfont}   % use normal font for page header
\ohead*{\thepage}                       % page number
\ihead*{\leftmark}                      % chapter name
\setlength{\headheight}{18pt}           % set headheight to 18pt
\addtolength{\skip\footins}{0.5cm}      % spacing at the end of page (footnotes)
\usepackage[bottom]{footmisc}           % Always place footer at end of page

\usepackage{multicol}                   % Format text in multiple columns
\usepackage{emptypage}                  % Support for empty pages via \cleardoublepage
\usepackage{lscape}                     % Support for landscape pages

% ##################################################
% IMAGES AND FIGURES
% ##################################################
\usepackage{graphicx}                   % support for including images
\graphicspath{{img/}}                   % default path
\usepackage{subfig}

% simple numbering without chapter
\renewcommand{\thefigure}{\arabic{figure}}

% ##################################################
% TikZ | Support for drawing diagrams
% ##################################################
\usepackage{tikz}
\usepackage{float}
\usetikzlibrary{shapes.geometric, arrows, positioning, decorations.pathreplacing, calc}
\tikzstyle{box} = [rectangle, minimum width=3cm, minimum height=1cm, text centered, draw=black, fill=orange!30]
\tikzstyle{mainbox} = [rectangle, minimum width=3cm, minimum height=1cm, text centered, draw=black, fill=green!30]
\tikzstyle{plainbox} = [rectangle, minimum width=3cm, minimum height=1cm, text centered, draw=black, fill=white,thick]
\tikzstyle{border} = [rectangle, minimum width=3cm, minimum height=1cm, text centered, draw=black]
\tikzstyle{arrow} = [thick,->,>=stealth]

\pgfdeclarelayer{foreground}
\pgfdeclarelayer{background}
\pgfsetlayers{background,main,foreground}

% ##################################################
% MATH
% ##################################################
\usepackage{amstext}
\usepackage{amsmath}

\usepackage{tabularx}
\newenvironment{conditions*}
  {\par\vspace{\abovedisplayskip}\noindent
   \tabularx{\columnwidth}{>{$}l<{$} @{${}={}$} >{\raggedright\arraybackslash}X}}
  {\endtabularx\par\vspace{\belowdisplayskip}}

% ##################################################
% SUPPORT FOR DIRECTORY TREE RENDERING
% ##################################################
\usepackage{dirtree}

% ##################################################
% SUPPORT FOR CODE LISTINGS
% ##################################################
\usepackage{listings}
\usepackage{color}
\usepackage{xcolor}

\definecolor{backcolour}{rgb}{0.95,0.95,0.95} % Custom color for code background
\definecolor{codegreen}{rgb}{0,0.6,0}         % Custom color for comments in your code
\definecolor{codegray}{rgb}{0.5,0.5,0.5}      % Custom color for numbers in your code
\definecolor{codepurple}{rgb}{0.58,0,0.82}    % Currently unused

\renewcommand{\lstlistingname}{Code Snippet} % Here you can change the code listing name

\lstdefinestyle{normal}{
    backgroundcolor=\color{backcolour},
    commentstyle=\color{codegreen},
    keywordstyle=\color{magenta},
    numberstyle=\tiny\color{codegray},
    stringstyle=\color{orange},
    basicstyle=\ttfamily\small,
    breakatwhitespace=false,
    breaklines=true,
    captionpos=b,
    keepspaces=true,
    numbers=none,
    numbersep=5pt,
    showspaces=false,
    showstringspaces=false,
    showtabs=false,
    tabsize=2,
    rulecolor=\color{black},
    frame=L,
    xleftmargin=6px
}
\lstset{style=normal}